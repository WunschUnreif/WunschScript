\chapter{前言}

WunschScript\footnote{\url{https://github.com/WunschUnreif/WunschScript}}这门编程语言是我在COVID-19疫情爆发期间为打发时光而编写的,其本意只是希望锻炼语言解释器的实现技术,并非为了解决任何实际问题。这门语言在语法上大量借鉴了JavaScript,而在语义上则有所不同,例如,采取了强类型机制,并引入了弱引用声明以替代垃圾回收(实际上只是懒得写GC)。此外,我还将一部分HTML的语法吸收进WunschScript中,例如通过尖括号进行对象的深拷贝,并可在其中加入键值对,以修改或扩充对象的内容,从而使程序显得更加紧凑,便于编写。尽管这门语言在实际生产中还没有用武之地,但对于简单的算法原形的实现,或实用小工具的开发,以及简单的计算任务,仍可以作为一门不错的工具语言。

\par 
在这本书中,我将通过大量的实例,由浅入深地全面展示WunschScript的语法以及各种特性,

\par 

WunschScript目前仍处于快速的发展当中,它将在未来引入更多方便的特性,运行速度也将逐渐优化。当然,它作为一种玩具语言的地位想必会长期保持下去。不过,对于初学编程或想要了解程序语言背后机制的人来说,WunschScript应当是一个不错的选择。
